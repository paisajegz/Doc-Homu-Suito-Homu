\section{materiales utilizados}
\subsection{Node.js}
es un entorno en tiempo de ejecución multiplataforma, de código abierto, para la capa del servidor basado en el lenguaje de programación JavaScript.\\
En node utilizaremos:
\subsubsection{Express}
\begin{itemize}
\item Es un marco de aplicación web para Node.js.
\item Está diseñado para crear aplicaciones web y API. Se le ha llamado el marco de servidor estándar de facto para Node.js.
\end{itemize}
\subsubsection{Electron}
Es un framework actualmente sostenido por github, para crear aplicaciones de escritorios con tecnología web (JavaScript, HTML, CSS).
\subsubsection{Mongoose}
Es un ODM (Object Document Mapped) El cual mapea la base de datos noSQL  tomando en cuenta sus documentos, y facilita el uso de la base de datos en el software.
\subsubsection{Socket.io}
es una biblioteca de JavaScript para aplicaciones web en tiempo real. Permite la comunicación bidireccional en tiempo real entre clientes y servidores web. 
\subsubsection{mqtt}
Es una librería la cual se utiliza para crear la conexión con el protocolo mqtt el cual se manejara en mosquitto
\subsubsection{JWT}
Es un framework de autenticación, que hoy es día es usado para hacer la validación de seguridad dentro de los servicios web de cada aplicación, programa o pagina web.
\subsubsection{Mcrypt}
Es una librería con múltiples algoritmo de encriptacion y desencriptacion, y es muy utilizado para poder ocultar la información y poder decodificar la información para poder leerla en los momentos adecuados.
\subsection{Mosquitto}
Eclipse Mosquitto es un agente de mensajes de código abierto (con licencia EPL / EDL) que implementa el protocolo MQTT versiones 5.0, 3.1.1 y 3.1. Mosquitto es liviano y es adecuado para usar en todos los dispositivos, desde computadoras de una sola placa de baja potencia hasta servidores completos.
